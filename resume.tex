\documentclass[uplatex, a4paper, 12pt]{jsarticle}
\usepackage[top=30mm, bottom=30mm, left=25mm, right=25mm]{geometry}
\usepackage{setspace}

\setstretch{1.3}  % 行間をやや広くする

\begin{document}

% --- 表紙 ---
\thispagestyle{empty} % ページ番号を非表示

\begin{center}
\vspace*{2cm}

{\LARGE 2025 年度 芝浦工業大学 工学部 情報工学科}

\vspace{1.5cm}

{\Huge \textbf{卒 業 論 文}}

\vspace{2.5cm}

{\Large 光るキーボードを用いた写経型学習による\\[0.8em]
ソースコードタイピング能力への影響の調査}

\vspace{3cm}

{\Large
学籍番号  AL22084\\[1em]
氏    名  中橋 哉斗\\[1em]
指導教員  笹埜 功
}

\vfill

{\Large 提出日  2025年○月○日}

\end{center}

\clearpage

% --- 第1章 ---
\section{はじめに}

2022 年に高等学校情報科が改定されるまで,高校の情報科では「社会と情報」,「情報の科学」のいずれか 1 科目を選択する方式だった.8 割の生徒はプログラミング学習がカリキュラムに含まれない「社会と情報」を履修し,残りの 2 割の生徒はプログラミング学習を行わずに高等学校卒業を迎える.

\end{document}